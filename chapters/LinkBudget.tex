\chapter{无线通信}

\subsection{描述无线通信的基本挑战}

我们想要尽可能优化以下元素:

\begin{itemize}
	\item 数据传输速率
	\item 更远的距离
	\item 在尽可能小的带宽
	\item 延迟
	\item 耗电量
	\item 尽可能低的错误率
	\item 尽可能小的设备和成本
\end{itemize}

\subsection{比较不同无线技术发射机的功率输出}

\subsection{在 dBi 和 dBd 之间转换天线增益}

在主要辐射/接收方向上,增益的测量单位是 dBi(各向同性的 dB)或 dBd(偶极子的 dB)。

dBd = dBi - 2.15 dB

\section{链路预算}

\subsection{计算给定波长和距离的 Friis 公式路径损耗,并转换为 dB}

\[\text{path loss(dB)} = 10 \log_{10}\frac{(4\pi r)^2}{\lambda^2}\]

\subsection{说明无线接收器的几个指标}

\begin{itemize}
	\item 工作的载波频率(必须与发射机的频率一致)
	\item 接收带宽
	\item 灵敏度(取决于带宽和本底噪声noise floor)
	\item 调制方案(modulation scheme)
\end{itemize}

\subsection{解释噪音、干扰和衰减之间的区别}

噪音是接收器由于物理特性导致的。噪声功率为$P = kT\Delta f$。

干扰是指与所需信号频率相近的其他信号对通信的影响,由人工干扰源和自然干扰源造成。

衰减是由于信号从发射器通过多条路径到达接收器,并部分抵消(cancel out)。

\subsection{计算给定温度下给定带宽的本底噪声(考试中提供了玻尔兹曼常数)}

噪声功率为$P = kT\Delta f$。

\subsection{说明衰减在移动通信中是如何发生的}

衰减是由于信号从发射器通过多条路径到达接收器,并部分抵消(cancel out)。信号的反射(reflection)和/或折射(refraction)可能导致多个路径。

\subsection{使用香农-哈特雷容量定理,计算在给定带宽内,指定信道容量所需的最小信噪比}

由于香农-哈特雷定理:

\[C = B \log_2 (\frac{S + N}{N})\]

所以,如果我们想在带宽 $B$ 中以一定的速率 $R\leq C$ 进行通信,这意味着我们需要一定的最小信噪比:

\[\frac{S + N}{N} \geq 2^{\frac{R}{B}}\]

\subsection{在指定发射机功率、天线增益、路径损耗等的情况下,编制并评估特定场景下的链路预算}

\subsection{解释在有替代组件可供选择的情况下,您是否/如何 ``修复''断开的链接预算}


\chapter{移动网络}

\section{移动设备}

\subsection{说明基于 802.11 的 WiFi 中的移动节点如何在同一 WiFi 网络的基站之间移动}

移动节点与AP1的互动:

\begin{itemize}
	\item 移动节点首先发送探测请求(Probe)。
	\item AP1回应一个探测响应(Probe Response)。
	\item 随后,移动节点发送关联请求(Association Request)。
	\item AP1随后发送关联响应(Association Response),此时移动节点与AP1建立了连接。

\end{itemize}

移动节点从AP1到AP2的迁移:

\begin{itemize}
	\item 当移动节点移动到AP2的范围内并尝试与AP2连接时,它首先发送一个探测请求。
	\item AP2回应一个探测响应。
	\item 移动节点发送关联请求,AP2随后回应关联响应,此时移动节点与AP2建立了连接。
	\item AP2通过分布系统(Distribution System)通知AP1关于关联变化。这确保了网络的连续性和流畅的转移。
	\item AP1接收到解除关联信息,知道移动节点已经不再与其连接。
\end{itemize}

\subsection{说明PPP的目的}

PPP被广泛用作许多点对点链接的底层协议。可以将更高层次的协议数据包封装在PPP数据包中,然后通过这个链接传输。这意味着PPP可以为其他协议(如IP)提供一个通信渠道。

在虚拟电路上运行PPP协议,这样移动节点和IP网关之间就好像只有一条连接链路。尽管在移动节点和IP网关之间可能有多个接入点,但使用PPP和虚拟电路,它们看起来就像是直接连接的。

\section{蜂窝网络}

\subsection{说明基于突发的通信是如何工作的(如在 GSM 中),以及为什么用户设备的上行链路和下行链路的突发时隙在时间上是分开的}

这是GSM对时分多路访问(TDMA)的实现。在每个200 kHz子频带上,有8个burst)。每个突发持续约0.577 ms并包含156.25位。8个burst组成一个完整的帧,持续4.615 ms。这意味着,在同一时间点,同一频率的一个子频带可以服务于8个不同的用户,因为他们是在不同的burst中传输数据的。

上行和下行通道之间的3个突发间隔意味着移动设备在发送和接收数据时永远不会发生冲突或干扰。这个设计考虑了移动设备的硬件限制,确保它们在同一时间只进行发送或接收操作。

\subsection{说明基于burst的移动通信中的突发帧如何限制通信距离,以及如何扩展这一限制}

burst传输可以被延迟最多8个比特。这意味着约有9公里的往返距离。如果距离超过这个限制,那么从移动设备返回的burst会干扰随后的burst。这可能导致数据传输的丢失或干扰。

为了扩展这个距离,可以留出一些burst不使用,从而为接收提供更多的时间。这种方法可以帮助减少由于距离过长而导致的信号干扰。

\subsection{描述蜂窝网络基站选址如何随着网络的发展而变化}

网络发展初期用户少、基站少、频率资源压力小、小区大。小区基站主要建在山顶(hilltops)、大型建筑物上,以最大限度地扩大每个小区的覆盖范围。

网络发展的成熟阶段用户多、站点多、频率资源压力大、小区小。小区基站大多位于山谷(bottom of valleys)底部的地面上--利用山丘或建筑物作为屏障,限制小区的覆盖范围/用户数量,并使山丘另一侧能够重复使用频率

\subsection{描述为快速移动的火车和飞机提供移动服务所面临的挑战}

如果手机相对于基站快速移动,就会出现一系列问题:

\begin{itemize}
	\item 多普勒效应(Doppler effect)会导致快速的多径衰落,也会改变接收到的物理比特率
	\item 多个基站之间的频繁切换会给系统带来很大负荷
	\item 接力(handover)要复杂得多
\end{itemize}

\subsection{说明为什么不能在飞机上使用手机}

如果在机场附近的地面上使用,某些 5G 频率确实会造成问题。

\subsection{说明 TCH、FACCH、SACCH、RACH、AGCH 和 PCH 在移动系统(如 GSM)中的作用}

26 个 GSM 帧连接成一个多帧,其中:

\begin{itemize}
	\item  2 x 12 个帧携带通信信道 (TCH),其中一个或多个可由快速相关控制信道 (FACCH) 帧取代
	\item 1 个帧是慢速相关控制信道 (SACCH),用于传输与 FACCH 相同的信息,但紧急程度较低
	\item 1 帧为未使用帧,此时所有基站都关闭,移动电话可以监听其他基站,查看周围是否有更强的基站。	也可用于其他各种控制和广播信道:
		\begin{itemize}
			\item RACH(随机接入信道),移动电话用于请求网络接入
			\item AGCH(接入许可信道),基站用来告诉移动电话它已被接受接入网络,以及使用哪些参数
			\item PCH(寻呼信道),用于在有来电时呼叫移动电话
			\item 还可以作为同步和频率参考信道
		\end{itemize}
\end{itemize}

\subsection{举例说明移动网络如何跟踪在不同供应商之间漫游的手机}

\subsubsection{海外漫游的本地号码呼叫本地号码}

手机会与基站建立连接。基站知道手机在其覆盖范围内。

\textbf{基站与访客交换中心 (MSC):}基站将手机的相关信息告知访客交换中心。此例中,该交换中心位于澳大利亚。

\textbf{访客交换中心与国际电话/网络:}访客交换中心将手机信息转发至国际电话/VoIP网络。

\textbf{国际电话/网络与主注册中心 (HLR):}国际电话/VoIP网络从访客交换中心接收信息,并传递给主注册中心或Gateway MSC,如新西兰的沃达丰网络。

\subsubsection{本地号码呼叫海外漫游的本地号码}

\textbf{通话经由国际电话网络:}其他手机发起的通话首先通过国际电话网络,然后路由至当前手机所在位置的MSC。

\textbf{主注册中心的角色:}通话也经由主注册中心的GMSC。

\textbf{MSC接收通话:}访客交换中心(MSC)受国际电话网络通知,在澳大利亚接收到这个通话,并将其路由至相应的基站。

\textbf{基站与手机通信:}基站将通话连接到目标手机。


\subsection{解释移动电话如何使用相同的 IP 地址与互联网通信,即使该地址在基站之间接力}

移动网络运营商首先使用移动协议建立虚拟数据电路到其互联网网关。移动节点和网关使用 PPP 或类似协议建立链接,通过 PPP 链路与移动提供商协商 IP 配置。由于节点移动时,虚拟电路和 PPP 连接保持不变,所以节点在运营商网络内移动时,基站之间的移动仅由 PPP 以下的移动协议处理,IP 配置一般保持不变。

\subsection{解释使移动电话技术中的频谱得到有效利用的三个关键概念}

\subsubsection{双工}

如何确保用户可以同时收听和通话。分为TDD和FDD。

\subsubsection{多重访问}

如何在一块频谱上为多个用户提供服务。分为FDMA,TDMA,CDMA,OFDMA或以上的混合方法。

\subsubsection{避免传输中出现符号错误}

通常需要远离信道容量。

另一种方法是:以高传输速率进行传输,并接受会出现错误的事实,但通过强大的纠错功能进行补偿。

\section{5G与物联网}

\subsection{说明为什么 5G 是通过信号电缆连接传感器和执行器的一种有吸引力的替代方案}

大量传感器/执行器的布线成本高昂,并且由布线错误和损坏风险。查找故障页相当费力。电缆含有昂贵的金属,造价较高。信息源和信息接收放之间的距离不是直线。另外,电缆延迟比无线电延迟高50\%。

\chapter{卫星网络}

\subsection{说明卫星频段的变化(如从 C 频段到 Ku 或 Ka 频段)对路径损耗的影响,以及对可能要使用的地面站天线的尺寸、增益和方向性的影响}

\textbf{L 波段(1-2 千兆赫):}
\begin{itemize}
	\item 最适合窄带应用的小波段
	\item 不受雨衰影响
	\item 不需要碟形天线
	\item 带宽昂贵
\end{itemize}

\textbf{C 波段(4-8 千兆赫):}
\begin{itemize}
	\item 适合宽带应用
	\item 需要大型天线
	\item 不受雨衰影响
	\item 但会受到微波干扰
\end{itemize}

\textbf{X 波段(9-12 千兆赫):}
\begin{itemize}
	\item 特性类似于 C 波段
	\item 保留给政府/军队
\end{itemize}

\textbf{Ku 波段(12-18 千兆赫):}
\begin{itemize}
	\item 可用于宽带应用
	\item 使用较小的碟形天线
	\item 雨衰可能导致问题
\end{itemize}

\textbf{Ka 波段(26.5-40 千兆赫):}
\begin{itemize}
	\item 适用于使用较小碟形天线的宽带应用
	\item 带宽便宜
	\item 雨衰问题严重
	\item 需要非常精确地指向 GEO 碟形天线
\end{itemize}

\textbf{E 波段(60-90 千兆赫):}
\begin{itemize}
	\item 天线小
	\item 信号表现与可见光相似
	\item 带宽便宜
	\item 大气气体造成严重衰减
\end{itemize}

越往后,需要的天线越小,衰减越大,方向性越强。

\subsection{说明``弯管''在卫星网络中的含义,以及它如何限制低地轨道网络的覆盖范围}

一个卫星连接两个地面站的示意图。其中一个地面站(标记为Gateway)连接到Internet,而另一个地面站(标记为ISP network)则为IP主机提供服务。

低地球轨道卫星通常不会一直位于地面网关站的信号覆盖范围内。由于LEO卫星不能一直保持与特定的地面站连接,因此需要在卫星之间进行数据传输的路由。这样可以确保数据能够在多个卫星之间转发,最终达到其目的地。路由协议需要动态适应卫星的位置变化。

\subsection{解释低地轨道星座的轨道倾角如何影响覆盖范围}

并非地球上的所有区域都能由特定的卫星轨道覆盖。一个卫星轨道的覆盖范围依赖于它的倾角(相对于地球赤道的角度)。一些轨道可能设计得更靠近赤道,而其他的可能包含了更高的纬度。因此,覆盖范围取决于卫星的轨道倾角。

\section{TCP队列震荡}

\subsection{描述 TCP 队列振荡的四个阶段(队列填充水平、输入流量速率),以及各阶段如何相互影响}

\textbf{卫星网关队列未满:}

这是当TCP发送方接收到确认消息(ACKs)时的起始阶段。
在这个阶段,由于接收到了确认消息,TCP发送方会增加其拥塞窗口。这意味着它会增加其发送的数据量。
随着更多数据被发送到队列,队列开始积累。

\textbf{卫星网关队列满:}

当队列满时,新到达的数据包将被丢弃。
尽管数据包被丢弃,但发送方仍然接收到确认消息。因此,他们继续发送更多的数据。
这导致队列持续超载,结果是突发性的数据丢失,也被称为“burst losses”。

\textbf{来自已丢弃数据包的确认消息变得过期:}

在某个时刻,由于数据包丢失,确认消息会变得过期。
这导致发送方开始限制其数据发送速率。具体来说,他们减少发送的数据量。
随着数据发送速度减慢,队列开始排空。

\textbf{队列完全清空:}

在这个阶段,由于发送方大大减少了其发送的数据量,队列完全被清空。
结果是链接在某些时段处于空闲状态,也就是说,它没有被充分利用。

\subsection{描述 TCP 队列振荡对大容量下载和接收简短电子邮件的影响}

邮件标题和邮件信息加载速度很快,但大型附件加载速度却不快。

一般来说,小文件可以快速加载,大文件则很慢。

\subsection{描述如何识别瓶颈卫星链路上的 TCP 队列振荡}

卫星链路的带宽有限,当多个发送方试图发送大量数据时,这些数据在卫星网关的队列中积累,形成瓶颈。队列可能会在完全空和完全满之间快速摆动。

\subsection{解释瓶颈卫星上的总流量需求如何决定是否出现 TCP 队列振荡,并解释在哪些情况下没有 TCP 队列振荡不是好消息}

当多个发送方试图发送大量数据时,这些数据在卫星网关的队列中积累,形成瓶颈。

如果没有队列震荡,可能意味着链路的容量并未得到充分利用。

\subsection{解释什么是卫星间链路}

在同一轨道平面(same orbital plane)(通常是同一轨道)上的卫星之间的联系。

\subsection{说明低地轨道网络路由选择具有挑战性的原因。举例说明困难的原因,以及在此过程中可以利用低地轨道网络的哪些特性}

\subsubsection{LEO卫星的体积和重量}

LEO互联网卫星通常比地球静止轨道(GEO)卫星更小、更轻。典型的GEO卫星重达数吨,而典型的中地球轨道(MEO)通信卫星可能只有几百千克。这意味着LEO卫星需要较少的能量来进行通信操作,但同时也意味着它们对电池和太阳能板的重量有更严格的限制。在设计时必须考虑如何平衡能量供应和设备重量的问题,这对LEO卫星的功率预算是一个关键考虑因素。

\subsubsection{充电周期}

卫星只能在能“看到”太阳时充电。对于GEO卫星,每天只有一次充电和放电的循环,因为它几乎全天候(大约22小时45分钟)都能看到太阳。而LEO卫星由于距离地球较近,它们绕地球的速度更快,因此一天中会经历约15次充电和放电的循环,大约每天只有12小时的时间能看到太阳。

\subsubsection{电池寿命}

电池的寿命是按充放电循环来衡量的。考虑到LEO卫星一天要经历多次充放电循环,这意味着它们的电池寿命可能比GEO卫星的短很多。因此,LEO卫星的功率预算不仅要考虑如何保持卫星在运行周期内的能量供应,还要考虑电池的更换和维护问题。



\chapter{星链}

\subsection{解释 "星链 "用户终端如何在给定纬度的地点(如奥克兰或赤道上)定位,以及这样做的原因}

\subsection{解释为什么星链不会取代人口稠密地区的地面互联网}

星链仅纬度 40 度左右至 53 度之间的覆盖范围大,赤道地区卫星密度较低。

就算不考虑高峰期负载,也需要非常多的卫星才能满足上网需求。而且许多卫星只能看到很少或根本看不到用户。

\subsection{请解释在什么情况下,您会期望从星链获得高数据速率的覆盖}

\subsection{解释什么是地球静止轨道保护以及为什么需要它}

如果地面站位于地球静止弧线 18° 以内,则不得向低地轨道卫星发射。

低地轨道地面站会干扰位于目标低地轨道卫星 "后方 "的地球同步轨道卫星。

\subsection{解释``direct‐to‐site''的含义,以及为什么使用直接到站点连接模式的卫星网络会给内容传输带来挑战}

一个卫星直接连接到多个用户站点的示意图。从Internet通过Gateway发送的数据通过卫星直接传输到各个IP主机。

需要每位用户的天线能够以自动方式跟踪卫星。


\chapter{海底光缆}

\subsection{描述电缆船如何修复断裂的海底电缆,以及 CD 和 HD 基本操作的原理}

\begin{itemize}
	\item \textbf{使用Rov或CD(低可见度)在受损地点附近切割线缆}
	\item \textbf{在受损地点2km之外使用HD回收两端电缆,}对回收的第一根缆线端头进行检查、密封并绑在浮标(buoy)上
	\item \textbf{将第二根缆线端头与备用缆线接头拼接,因为电缆末端在海面上会有一定距离}
	\item \textbf{返回第一个浮标,将备用缆线与浮标上的缆线端头连接起来}
    \item \textbf{将缆线在海底绕成一圈,以防打结}
\end{itemize}

整个过程需要2个HD和最多1个CD。

CD:切割驱动(拖曳)

HD:保持驱动(拖拽)

\chapter{TCP与拥塞控制}

\section{拥塞窗口}

\subsection{说明慢启动和 AIMD 之间的区别。解释为什么它们的组合对在高 BDP 链路上运行的 TCP 发送机不起作用}

慢启动是TCP连接开始时用来快速增加发送方的发送速率的方法。它的名称“慢启动”有些讽刺,因为增长是指数级的。具体来说:

\begin{itemize}
	\item 开始时,cwnd从一个较小的值(比如10)开始。
	\item 对于每个收到的ACK,cwnd增加1,导致每个RTT cwnd翻倍。
	\item 这个过程一直持续到达到慢启动阈值(ssthresh,可以为连接带宽的一半)或遇到丢包事件。
\end{itemize}

达到慢启动阈值ssthresh后,或在遇到丢包事件后,TCP使用AIMD来调整cwnd,以避免造成网络拥塞。AIMD的具体规则包括:

\begin{itemize}
	\item 加法增加:在每个没有丢包的RTT结束后,cwnd只增加一个固定的量(比如1)。

	\item 乘法减少:在检测到丢包事件时,cwnd将以乘法的方式减少,比如减少为原来的0.7或0.5。
\end{itemize}


在具有高带宽延迟乘积(BDP)的链路上,慢启动和AIMD的组合可能并不理想,原因如下:

\begin{itemize}
	\item AIMD 增长缓慢--在高 BDP 链路上效率低下
	\item 与拥塞无关的零星损失后过度回退
\end{itemize}

\subsection{解释为什么 WiFi 接口比现代以太网接口更容易出现基于主机的拥塞}

在发送主机上,当操作系统内核向接口发送的数据超过接口向介质推送的能力时拥塞会发生。

WiFi接口向介质推送的能力不如以太网接口。此外,WiFi链路是一个共享介质,当WiFi信道繁忙时会相互干扰,导致数据包延迟或丢失。

\section{缓冲区管理}

\subsection{描述 PFIFO 和 BFIFO 的区别,并说明每种队列规则的优缺点}

数据包FIFO是指FIFO的最大容量以数据包数量计;
字节FIFO是指FIFO的最大容量以字节计。

PFIFO优点:

\begin{itemize}
	\item 实现简单:PFIFO可以通过SKB的链表实现。SKB(Socket Kernel Buffers):SKB是可回收利用的缓冲区,用于存放任意大小的单个数据包,因此处理起来非常快速。
\end{itemize}

PFIFO缺点:

\begin{itemize}
	\item 队列字节量不明确:当队列满时,很难准确知道队列中有多少字节的数据,因为每个数据包的大小可能不同。
	\item 队列停留时间波动:即使以恒定速率出队数据包,由于数据包大小不一,队列的停留时间(即从队列中清除所有数据包所需的时间)也会有很大差异。

\end{itemize}

BFIFO优点:

\begin{itemize}
	\item 与带宽延迟乘积BDP相关: BFIFO的字节限制使得它更容易与链路的带宽延迟乘积(即链路可以在传输延迟期间持有的数据量)比较,这有助于更好地管理缓冲。
	\item 实现相对简单: 从技术上讲,实现BFIFO与PFIFO类似,只是需要考虑数据包的大小。
\end{itemize}

BFIFO缺点:

\begin{itemize}
	\item 资源利用率低: 如果使用SKB(Socket Kernel Buffers)来存储数据包,那么小数据包只会使用SKB容量的一小部分。如果需要排队的数据包很多,就需要大量的SKB,这些SKB大部分时间都是空的,这导致了内存资源的浪费。
	\item 可能的不公平性: 当缓冲区快满时,如果一个大的IP数据包到达(例如1500字节),但后面跟着多个小数据包(例如TCP的SYN或ACK包,每个包40字节),大的数据包可能因为空间不足被丢弃,而小的数据包却能够入队。这在某些情况下可能会导致不公平。


\end{itemize}

\subsection{如何测试已知延迟的瓶颈链路(不传输其他流量),以确定其是否配置了应有的缓冲容量和带宽}

\subsubsection{预先准备}

测试涉及两个实体,一个客户端和一个服务器。客户端位于链路的一侧,而服务器位于另一侧。客户端向服务器发送UDP数据包,数据包中包含:

\begin{itemize}
	\item 序列号: 每个数据包都包含一个序列号,这使得客户端和服务器可以跟踪数据包,检测是否有数据包丢失。
	\item 微秒级时间戳: 数据包还包含一个时间戳,以微秒为单位,用于测量往返时间(RTT)和网络延迟。
	\item 填充: 数据包被填充到最大分段大小(即IP数据包大小等于最大传输单元MTU),这样做是为了确保每个数据包都有一致的大小,便于测试BFIFO或PFIFO。
\end{itemize}

服务器收到客户端的数据包后,会将其回送给客户端,但是去除了填充,这可能是为了减少回程数据包的大小,确保响应的迅速。

客户端使用序列号来确定是否有数据包在传输过程中丢失。如果收到的数据包序列号不连续,那么可以确定发生了数据包丢失;客户端使用时间戳来测量RTT,即数据包从发送到接收再回到发送所经历的总时间。

服务器端的输入缓冲区在这个测试设置中并不被测试,这可能是因为测试重点是瓶颈链路的客户端输入缓冲区。

\subsubsection{正式测试}

\textbf{第1阶段}

\textbf{操作:} 以链路的名义(nominal)速率R向链路发送数据。

\textbf{期望结果:} 往返时间(RTT)应该是2 * L,没有数据包丢失。

\textbf{分析:} 这表明链路可以在其名义速率R上正常运行而不丢失数据包。RTT的值确认了链路的基本延迟。

\textbf{第2阶段}

\textbf{操作:} 继续以速率R发送数据,但同时发送一个额外的数据包突发,大小为缓冲区B。

\textbf{期望结果:} RTT应该是2 * L + B / R。

\textbf{分析:} 这个阶段测试链路的缓冲能力,以及它能否在不丢失数据包的情况下处理短暂的流量增加。期望的RTT增加反映了数据通过填满的缓冲区所需的额外时间。

\textbf{第3阶段}

\textbf{操作:} 继续以速率R发送数据,并再次发送大小为B的数据突发。

\textbf{期望结果:} 应考虑数据包丢失和RTT的变化。

\textbf{分析:} 此阶段旨在测试缓冲区的恢复能力和链路处理连续高流量的能力。如果数据包丢失或RTT增加超过预期,则表明缓冲区不能及时恢复,可能太小,或者链路无法处理此类连续突发。

\textbf{第4阶段}

\textbf{操作:} 以略高于名义速率的1.1 * R的速率向链路发送数据。

\textbf{期望结果:} 不特别说明,但这个步骤可能测试链路在轻微超载情况下的表现。

\textbf{分析:} 通过超过链路名义速率的小幅度来观察缓冲区和链路的表现。如果链路和缓冲区能够处理这种轻微的超载而不出现显著的RTT增加或数据包丢失,这表明链路和缓冲区有一定的容错能力。如果出现问题,这可能指示链路的实际容量低于名义速率,或者缓冲区不足以处理超载。

\subsection{讨论 RED 如何工作,并给出 RED 工作必须满足的标准}

\subsubsection{工作原理}

\begin{itemize}
	\item 基于FIFO的操作: RED在基本原则上依然是按照FIFO(无论是BFIFO还是PFIFO)来操作的,即数据包通常按照到达的顺序进行处理和转发。
	\item 提前丢包: RED的关键特点是它会在缓冲区完全填满之前开始丢弃数据包。这是一种预防性的措施,目的是避免队列长度增加到会引起拥塞的程度。丢包的概率由以下因素决定:
		\begin{itemize}
		\item 队列长度阈值: 低于最小队列长度阈值时,接受所有数据包
		\item 超过该阈值时,数据包被丢弃的概率会随着队列长度达到最大队列容量而增加到1
		\end{itemize}
\end{itemize}

基于RED的目标,backoff对路由器 FIFO 到达率的影响必须在队列满之前产生影响。这就要求 RTT<填满队列所需的时间。对于大的 RTT 较难实现(大多数路由器会看到各种 RTT)。所以我们鼓励使用更大的缓冲区


\subsection{描述如果我们为路由器增加越来越多的缓冲存储器,以防止因路由器队列尾部掉线而造成数据包丢失,会发生什么情况。给出支持和反对大型路由器缓冲区的理由}

\subsubsection{支持}

\begin{itemize}
	\item 增加带宽延迟积(BDP)
	\item 激励TCP增大其拥塞窗口
\end{itemize}

\subsubsection{反对}

\begin{itemize}
	\item 如果部署了更多的缓冲内存,实际上我们通过增加延迟来增加BDP。更多的缓冲内存意味着更长的队列逗留时间。更长的队列逗留时间会导致更大的往返时间(RTT)。而更大的RTT意味着拥塞窗口的增长会变慢。
	\item 没有区分队列:时间关键型数据包被卡在队列中,排在非时间关键型数据包后面
\end{itemize}

\subsection{描述传统路由器缓冲区尺寸与 Appenzeller 建议之间的区别,并讨论实施Appenzeller建议需要了解的路由器流量信息}

传统上,网络缓冲区的大小被设置为相等于裸链路的带宽延迟乘积(BDP),即未经过任何形式增强的连接链路的BDP。这种方法简单明了,因为裸链路的BDP容易计算,而且这种方法至今仍被广泛使用。

然而,Guido Appenzeller在2004年提出,传统方法中使用的BDP实际上过大。他建议应该使用BDP除以根号N,其中N是长期TCP流的数量(即持续时间超过1个RTT的TCP流)。这个理论基于这样一个事实,即随着长期TCP流数量的增加,每个流由于共享带宽而实际需要的缓冲区大小会减少。

为了实施Appenzeller建议,我们需要知道BDP和长期TCP流的数量。

\subsection{说明``缓冲膨胀''一词的含义}

现代网络中的缓冲区通常过大,这会导致延迟和网络性能问题。

这个观察随后促成了一些解决方案的开发,例如codel、fq\_codel和cake这些算法。这些算法的目的是减少延迟,它们通过动态调整网络队列的处理方式来控制和减少缓冲膨胀。

\section{ECN}

\subsection{说明 ECN 的工作原理,以及 TCP 发送方、接收方主机和路由器在其中扮演的角色}

ECN工作原理是使用IPv4或IPv6头部中的服务类型(TOS)/流量类别字段的最后两位。为了使用ECN,发送和接收的IP主机都必须启用ECN,网络中的路由器也必须支持并启用ECN。

ECN设置:具有ECN功能的发送端主机会在IP头部的TOS/流量类别字段中设置其中一个位为1,表示其数据包是ECN能够识别的。

在无拥塞时:如果路由器没有遇到拥塞,它不会改变这两个位。

在拥塞时:如果路由器经历拥塞,并且检测到一个非零的ECN位,它会设置另一个位为1,这相当于是一个拥塞通知标记。

通知传递:接收端主机接收到这个标记后,会将这个拥塞通知传递给传输层,例如TCP。

如果接收端主机在IP报头中检测到ECN设置的位为11(表示遇到拥塞),它将在TCP报头的标志字段中设置ECN-Echo (ECE) 标志位为1,并在发送回发送端的TCP报文中标记此位。接收到含有ECE标志位的TCP报文的发送端主机会减小其拥塞窗口(cwnd),并在其回复的TCP报文中设置拥塞窗口减少(CWR)标志位。

在TCP层,通过设置ECE和CWR标志位,ECN能够在不丢弃数据包的情况下传递网络拥塞的信号。这与因为丢包而启动拥塞控制机制相比,是一种更为温和的响应方式。这样可以避免因为路由器缓冲区溢出而导致的数据包丢失,避免了因过度反应而导致的性能下降。

\section{RTT分布}

\subsection{解释为什么不同的 RTT 分布会帮助或阻碍拥塞控制}

如果 RTT 分布大致均匀:

\begin{itemize}
	\item 所有的TCP发送者对于数据包丢失(无论是尾部丢包、RED,还是其他形式)的响应时间框架都大致相同,所有流量或多或少在同一时间发生丢弃
	\item 所有长流量同时backoff
	\item 全局同步(TCP 队列振荡,TCP queue oscillation)所有流同时增加和减少它们的拥塞窗口,造成整个网络吞吐量的剧烈波动
\end{itemize}


如果 RTT 分布不均匀:

\begin{itemize}
	\item TCP 发送方会在不同时间做出响应,有的立刻backoff,有的持续一段时间
	\item 一些发送方比其他发送方更快从backoff中恢复
	\item 这有助于拥塞控制
\end{itemize}












































