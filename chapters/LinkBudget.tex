\chapter{无线通信}

\chapter{蜂窝网络}

\chapter{卫星网络}

\chapter{星链}

\chapter{海底光缆}

\chapter{TCP与拥塞控制}

\section{拥塞窗口}

\subsection{说明慢启动和 AIMD 之间的区别。解释为什么它们的组合对在高 BDP 链路上运行的 TCP 发送机不起作用}

慢启动是TCP连接开始时用来快速增加发送方的发送速率的方法。它的名称“慢启动”有些讽刺,因为增长是指数级的。具体来说:

\begin{itemize}
	\item 开始时,cwnd从一个较小的值(比如10)开始。
	\item 对于每个收到的ACK,cwnd增加1,导致每个RTT cwnd翻倍。
	\item 这个过程一直持续到达到慢启动阈值(ssthresh,可以为连接带宽的一半)或遇到丢包事件。
\end{itemize}

达到慢启动阈值ssthresh后,或在遇到丢包事件后,TCP使用AIMD来调整cwnd,以避免造成网络拥塞。AIMD的具体规则包括:

\begin{itemize}
	\item 加法增加:在每个没有丢包的RTT结束后,cwnd只增加一个固定的量(比如1)。

	\item 乘法减少:在检测到丢包事件时,cwnd将以乘法的方式减少,比如减少为原来的0.7或0.5。
\end{itemize}


在具有高带宽延迟乘积(BDP)的链路上,慢启动和AIMD的组合可能并不理想,原因如下:

\begin{itemize}
	\item AIMD 增长缓慢--在高 BDP 链路上效率低下
	\item 与拥塞无关的零星损失后过度回退
\end{itemize}